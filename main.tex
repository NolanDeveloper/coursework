\documentclass [fontsize=14pt, paper=a4, pagesize, DIV=calc]%
{scrreprt}

\usepackage[T2A]{fontenc}
\usepackage[utf8]{inputenc}
\usepackage[russian]{babel}
\usepackage{indentfirst}
\usepackage{listings}
\usepackage{graphicx}
\usepackage{float}

\begin{document}

\begin{titlepage}

\centering

\vfill

МИНИСТЕРСТВО~ОБРАЗОВАНИЯ~И~НАУКИ~РФ

\vfill

Федеральное государственное автономное образовательное\\
учреждение высшего образования\\
ЮЖНЫЙ~ФЕДЕРАЛЬНЫЙ~УНИВЕРСИТЕТ

\vfill

Институт математики, механики и компьютерных наук\\
имени И.~И.~Воровича

\vfill

Направление подготовки Фундаментальная информатика\\
и информационные технологии

\vfill

\textsf
{МОДИФИКАЦИЯ МЕХАНИЗМА ОБРАБОТКИ ОТРИЦАТЕЛЬНЫХ ЛИТЕРАЛОВ В КОМПИЛЯТОРЕ GHC}

\vfill

Курсовая работа

\raggedleft

\vfill

Студента 3 курса\\
Д.~Ш.~Мирзоева

\vfill

Научный руководитель:\\
Старший преподаватель кафедры информатики и вычислительного эксперимента
Мехмата ЮФУ В.~Н.~Брагилевский

\end{titlepage}


\tableofcontents

\chapter*{Введение}
\addcontentsline{toc}{section}{Введение}

Во многих языках программирования присутствует понятие литерала. Литерал ---
это способ записи данных, неизменяемых в ходе работы
программы.\cite{cs-encyclopedia}

В данной курсовой работе нами рассмотрен язык Хаскель. В его стандарте
\cite{haskell2010} отсутствуют отрицательные литералы в качестве отдельной
лексемы.  Для записи отрицательных чисел используется унарная операция минус.
Такой подход порождает некоторые особенности, что побудило сообщество
разработчиков создать расширение для языка, добавляющее настоящие отрицательные
литералы.

Как выяснилось, способ, которым было реализовано данное расширение, не позволял
корректно обрабатывать значение ``отрицательный ноль''. \texttt{-0} компилятор
неверно распознавал как обыкновенный, положительный ноль. Такое поведение было
воспринято как ошибка. Нами были рассмотрены несколько путей решения данной
проблемы. Один из этих способов оказался приемлем для разботчиков компилятора.
Изменения подготовленные нами были приняты в основной репозиторий.


\chapter{Литералы в Haskell}

\section{Виды литералов}

В Хаскель можно выделить три вида литералов: целочисленные, вещественные,
символьные и строковые.

\begin{figure}[H]
\label{bool}
\begin{lstlisting}[language=Haskell]
data Bool = True | False
\end{lstlisting}
\caption{Определение значений True и False}
\end{figure}

В отличие от многих других языков, логические значения \texttt{True} и
\texttt{False} являются идентификаторами. Они входят в состав стандартного
модуля \texttt{Prelude} и определены как показано в листинге \ref{bool}.

\begin{figure}[H]
\centering
\includegraphics[scale=0.66]{pic-integral-literals}
\caption{Лексическая структура целочисленных литералов}
\end{figure}

Целочисленные литералы представляют значения натуральных чисел, включая ноль.
Существует синтаксис для записи в различных системах счисления: десятичной,
восьмеричной, шестнадцатеричной. Примеры: \texttt{30}, \texttt{0o36},
\texttt{0O36}, \texttt{0x1E}, \texttt{0X1E}.

\begin{figure}[H]
\centering
\includegraphics[scale=0.66]{pic-rational-literals}
\caption{Лексическая структура вещественных литералов}
\end{figure}

Вещественные литералы представляют рациональные числа. Можно использовать как
обычные десятичные дроби, так и научную форму записи с указанием мантиссы и
порядка. Примеры: \texttt{123.456}, \texttt{0.123456e3}.

\begin{figure}[H]
\centering
\includegraphics[scale=0.66]{pic-char-string-literals}
\end{figure}

Символьные литералы записываются в одинарных кавычках, а строковые в двойных.
Примеры: \texttt{'a'}, \texttt{"Hello, World!"}.



\section{Расширение NegativeLiterals}

\subsection{Описание}

Для решения проблем связанных с первыми двумя особенностями было создано
расширение NegativeLiterals. Оно добавляет в язык отрицательные литералы.
Чтобы использовать это расширение, необходимо либо добавить флаг компиляции
\texttt{-XNegativeLiterals}, либо добавить в начало файла с кодом программы
строку\\
\texttt{\{-\#\ LANGUAGE\ NegativeLiterals\ \#-\}}

\begin{ListingEnv}[H]
\begin{lstlisting}
$ ghci -XNegativeLiterals
GHCi, version 7.10.3: http://www.haskell.org/ghc/  :? for help
Prelude> print -1
-1
Prelude> import Data.Int
Prelude Data.Int> print (-128::Int8)
-128
\end{lstlisting}
\caption{Демонстрация NegativeLiterals}
\end{ListingEnv}

\subsection{Реализация}

Расширение модифицирует работу лексического анализатора. В его описание
добавляется несколько новых правил.

\begin{ListingEnv}[H]
\begin{verbatim}
@negative @decimal -- Регулярное выражение
    / { ifExtension negativeLiteralsEnabled } -- Условие
    { tok_num negative 1 1 decimal } -- Генератор лексемы
@negative 0[bB] @binary
    / { ifExtension negativeLiteralsEnabled `alexAndPred`
        ifExtension binaryLiteralsEnabled }
    { tok_num negative 3 3 binary }
@negative 0[oO] @octal
    / { ifExtension negativeLiteralsEnabled }
    { tok_num negative 3 3 octal }
@negative 0[xX] @hexadecimal
    / { ifExtension negativeLiteralsEnabled }
    { tok_num negative 3 3 hexadecimal }
...
@negative @floating_point
    / { ifExtension negativeLiteralsEnabled }
    { strtoken tok_float }
\end{verbatim}
\caption{Правила работы лексического анализатора, связанные с расширением
NegativeLiterals}
\label{lst:rules}
\end{ListingEnv}

Лексический анализатор компилятора GHC генерируется с помощью инструмента
Alex. В листинге \ref{lst:rules} приведены несполько правил, использующих
его синтаксис.

Каждое правило состоит из нескольких частей. Первая часть --- это регулярное
выражение, описывающее лексему. Здесь могут быть использованы макро
определения, то есть предварительно определённые регулярные выражения.  В
листинге \ref{lst:rules} такими макро определениями являются
\texttt{@negative}, \texttt{@decimal}, \texttt{@binary} и другие. Вторая часть
--- это контекст, в котором это правило может применяться. В приведённом
фрагменте проверяется включено ли расширение NegativeLiterals. Третья часть ---
это код на языке Хаскель, который создаёт объект лексемы на основе
последовательности символов, ей соответствующей.


\chapter[Проблема в расширении NegativeLiterals]
{Проблема обработки отрицательного нуля в расширении NegativeLiterals}



\begin{thebibliography}{9}

\bibitem{cs-encyclopedia}
Harry Henderson,
\emph{Encyclopedia of computer science and technology},
2003

\bibitem{haskell2010}
Simon Marlow(editor),
\emph{Haskell 2010 Language Report}

\end{thebibliography}

\end{document}
