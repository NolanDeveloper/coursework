\chapter*{Введение}
\addcontentsline{toc}{section}{Введение}

Во многих языках программирования присутствует понятие литерала. Литерал ---
это способ записи данных, неизменяемых в ходе работы
программы.\cite{cs-encyclopedia}

В данной курсовой работе нами рассмотрен язык Хаскель. В его стандарте
\cite{haskell2010} отсутствуют отрицательные литералы в качестве отдельной
лексемы.  Для записи отрицательных чисел используется унарная операция минус.
Такой подход порождает некоторые особенности, что побудило сообщество
разработчиков создать расширение для языка, добавляющее настоящие отрицательные
литералы.

Как выяснилось, способ, которым было реализовано данное расширение, не позволял
корректно обрабатывать значение ``отрицательный ноль''. \texttt{-0} компилятор
неверно распознавал как обыкновенный, положительный ноль. Такое поведение было
воспринято как ошибка. Нами были рассмотрены несколько путей решения данной
проблемы. Один из этих способов оказался приемлем для разботчиков компилятора.
Изменения подготовленные нами были приняты в основной репозиторий.
