\section{Расширение NegativeLiterals}

\subsection{Описание}

Для решения проблем связанных с первыми двумя особенностями было создано
расширение NegativeLiterals. Оно добавляет в язык отрицательные литералы.
Чтобы использовать это расширение, необходимо либо добавить флаг компиляции
\texttt{-XNegativeLiterals}, либо добавить в начало файла с кодом программы
строку\\
\texttt{\{-\#\ LANGUAGE\ NegativeLiterals\ \#-\}}

\begin{ListingEnv}[H]
\begin{lstlisting}
$ ghci -XNegativeLiterals
GHCi, version 7.10.3: http://www.haskell.org/ghc/  :? for help
Prelude> print -1
-1
Prelude> import Data.Int
Prelude Data.Int> print (-128::Int8)
-128
\end{lstlisting}
\caption{Демонстрация NegativeLiterals}
\end{ListingEnv}

\subsection{Реализация}

Расширение модифицирует работу лексического анализатора. В его описание
добавляется несколько новых правил.

\begin{ListingEnv}[H]
\begin{verbatim}
@negative @decimal -- Регулярное выражение
    / { ifExtension negativeLiteralsEnabled } -- Условие
    { tok_num negative 1 1 decimal } -- Генератор лексемы
@negative 0[bB] @binary
    / { ifExtension negativeLiteralsEnabled `alexAndPred`
        ifExtension binaryLiteralsEnabled }
    { tok_num negative 3 3 binary }
@negative 0[oO] @octal
    / { ifExtension negativeLiteralsEnabled }
    { tok_num negative 3 3 octal }
@negative 0[xX] @hexadecimal
    / { ifExtension negativeLiteralsEnabled }
    { tok_num negative 3 3 hexadecimal }
...
@negative @floating_point
    / { ifExtension negativeLiteralsEnabled }
    { strtoken tok_float }
\end{verbatim}
\caption{Правила работы лексического анализатора, связанные с расширением
NegativeLiterals}
\label{lst:rules}
\end{ListingEnv}

Лексический анализатор компилятора GHC генерируется с помощью инструмента
Alex. В листинге \ref{lst:rules} приведены несполько правил, использующих
его синтаксис.

Каждое правило состоит из нескольких частей. Первая часть --- это регулярное
выражение, описывающее лексему. Здесь могут быть использованы макро
определения, то есть предварительно определённые регулярные выражения.  В
листинге \ref{lst:rules} такими макро определениями являются
\texttt{@negative}, \texttt{@decimal}, \texttt{@binary} и другие. Вторая часть
--- это контекст, в котором это правило может применяться. В приведённом
фрагменте проверяется включено ли расширение NegativeLiterals. Третья часть ---
это код на языке Хаскель, который создаёт объект лексемы на основе
последовательности символов, ей соответствующей.
